\documentclass[a4paper]{article}
\usepackage[utf8]{inputenc}
\usepackage{geometry}
\geometry{a4paper}
\usepackage[polish]{babel}
\usepackage[QX]{fontenc}
\usepackage{lmodern}
\usepackage{indentfirst}
\usepackage{enumerate}
\title{Raport z implementacji algorytmu "Gra w życie"}
\author{Radosław Głombiowski}

\begin{document}

\maketitle

\tableofcontents

\section{Założenia projektu}
\subsection{Mechanika algorytmu}
Na planszy skonstruowanej z komórek zostają wyróżnione te komórki, które są żywe. Każda z komórek posiada ośmiu sąsiadów.
Po wprowadzeniu przez użytkownika pozycji żywych komórek następują kolejne etapy życia tych struktur.
Pozycje nowych komórek w nowej turze są obliczane według następujących reguł:
\begin{enumerate}
\item Jeśli martwa komórka posiada wokół siebie trzech żywych sąsiadów, komórka ożywa.
\item Jeśli żywa komórka posiada wokół siebie dwóch bądź trzech sąsiadów, komórka przeżywa.
\item Jeśli żywa komórka posiada wokół siebie inną liczbę sąsiadów, umiera (czy to z samotności, czy przeludnienia).
\end{enumerate}

\subsection{Cel projektu}
\begin{itemize}
\item Stworzenie programu sekwencyjnego i równoległego działającego na Xeon Phi opartych na algorytmie Gry w życie. Następnie porównać czasy wykonania obu programów i wyciągnąć z tego wnioski. 
\item Nauczenie się efektywnego programowania równoległego i programowania na co-procesor Xeon Phi.
\end{itemize}

\newpage

\section{Porównanie wersji sekwencyjnej i równoległej}

Testy zostały przeprowadzone na Sigmie oraz Xeon'ie Phi. Pomiary czasu zostały wykonane dla macierzy 1000x1000, 5000x5000, 10 000x10 000 oraz dla ilości kroków równej 1000. \newline
Sekwencyjny: \newline
\begin{tabular}{|l|r|r|r|} \hline
& 1000 & 5000 & 10000 \\ \hline
Tworzenie danych & 0.013487 & 0.325017 & 1.279214 \\ \hline
Mechanika & 28.756995 & 721.420322 & Unicestwiony \\ \hline
Całość & 28.770482 & 721.745339 & Unicestwiony \\ \hline
\end{tabular} \newline

Równoległy (8 rdzeni): \newline
\begin{tabular}{|l|r|r|r|} \hline
& 1000 & 5000 & 10000 \\ \hline
Tworzenie danych & 0.017524 & 0.208643 & 0.735874 \\ \hline
Mechanika & 6.267167 & 133.991264 & Unicestwiony \\ \hline
Całość & 6.284691 & 134.199907 & Unicestwiony \\ \hline
\end{tabular} \newline

Xeon Phi: \newline
\begin{tabular}{|l|r|r|r|} \hline
& 1000 & 5000 & 10000 \\ \hline
Tworzenie danych & 0.285410 & 0.726887 & 2.087383 \\ \hline
Mechanika & 2.670816 & 40.035989 & 173.218563 \\ \hline
Całość & 2.956226 & 40.762876 & 175.305946 \\ \hline
\end{tabular} \newline

Porównanie poszczególnych czasów (całość): \newline
\begin{tabular}{|l|r|r|r|} \hline
& 1000 & 5000 & 10000 \\ \hline
Sekwencyjny & 28.770482 & 721.745339& Unicestwiony\\ \hline
Równoległy (Sigma - 8 rdzeni) & 6.284691 & 134.199907 & Unicestwiony \\ \hline
Xeon Phi & 2.956226 & 40.762876 & 175.305946 \\ \hline
\end{tabular} \newline

Alokacja pamięci: \newline
\begin{tabular}{|l|r|r|r|} \hline
& 1000 & 5000 & 10000 \\ \hline
Sekwencyjny & 0.013487 & 0.325017 & 1.279214\\ \hline
Równoległy (Sigma - 8 rdzeni) & 0.056674 & 0.379938 & 0.841832 \\ \hline
Xeon Phi & 0.280319 & 3.816202 & 9.298919 \\ \hline
\end{tabular} \newline

\section{Wnioski}

Z porównań czasów jasno wynika, że tylko Xeon Phi poradził sobie z największym zadaniem jakim była macierz 10 000 x 10 000. Można z tego wywnioskować, że im większy problem stoi przed nami, tym bardziej będzie opłacalne zaopatrzenie się w ten sprzęt i jemu pokrewny. Nie można oczywiście zapominać, że samo posiadanie wielu rdzeni znacząco przyspiesza działanie programu. Jak można się przekonać działanie programu na sprzęcie wyposażonym w kilka rdzeni potrafi skrócić czas oczekiwania na wyniki nawet 5-cio krotnie. Z tego powodu każdy początkujący programista powinien uczyć się programowania równoległego już od samego początku. Wiedza ta będzie na pewno mile widziana w oczach pracodawcy. \newline
Jednakże należy uważać w trakcie prac nad wielordzeniową wersją programu. Mimo iż różne elementy wykonywane w trakcie uruchomienia programu intuicyjnie można by wykonywać na wielu rdzeniach, nie jest to wskazane. Takim przypadkiem jest np. alokacja pamięci. Biorąc pod uwagę ludzki tok myślenia jest to jedno z miejsc, które nadają się do wykonywania na wielu wątkach. Jednakże, jak wskazują wyniki, takie działanie jest błędne. Dzieje się to z powodu tego, iż alokacja pamięci i tak musi być wykonana sekwencyjnie, a każdy procesor, który chce zaalokować pamięć musi wiedzieć gdzie ją zaalokować co powoduje ''zamieszanie'' i komunikacyjny problem. Z tego powodu dostajemy spore opóźnienie w stosunku do wersji sekwencyjnej. Należy jednak zwrócić uwagę na pewną anomalię jaką jest szybsza alokacja na komputerze 8-io rdzeniowym. Przyczyny tej anomialii jednak są nieznane.

\section{Wiedza na przyszłość}

\begin{itemize}
\item Znajomość OpenMP i wiedza o krytycznych miejscach uniemożliwiających zrównoleglenie.
\item Wiedza jak pisać programy by zmaksymalizować użycie pamięci cache.
\item Umiejętność obsługi sprzętu jakim jest Xeon Phi.
\end{itemize}

\end{document}